%%%%%%%%%%%%%%%%%%%%%%%%%%%%%%%%%%%%%%%%%%%%%%%%%%%%%%%%%%%%%%%
%% OXFORD THESIS TEMPLATE

% Use this template to produce a standard thesis that meets the Oxford University requirements for DPhil submission
%
% Originally by Keith A. Gillow (gillow@maths.ox.ac.uk), 1997
% Modified by Sam Evans (sam@samuelevansresearch.org), 2007
% Modified by John McManigle (john@oxfordechoes.com), 2015
% Modified by Ulrik Lyngs (ulrik.lyngs@cs.ox.ac.uk), 2018, for use with R Markdown
%
% Ulrik Lyngs, 25 Nov 2018: Following John McManigle, broad permissions are granted to use, modify, and distribute this software
% as specified in the MIT License included in this distribution's LICENSE file.
%
% John tried to comment this file extensively, so read through it to see how to use the various options.  Remember
% that in LaTeX, any line starting with a % is NOT executed.  Several places below, you have a choice of which line to use
% out of multiple options (eg draft vs final, for PDF vs for binding, etc.)  When you pick one, add a % to the beginning of
% the lines you don't want.


%%%%% CHOOSE PAGE LAYOUT
% The most common choices should be below.  You can also do other things, like replacing "a4paper" with "letterpaper", etc.

% This one will format for two-sided binding (ie left and right pages have mirror margins; blank pages inserted where needed):
%\documentclass[a4paper,twoside]{templates/ociamthesis}
% This one will format for one-sided binding (ie left margin > right margin; no extra blank pages):
%\documentclass[a4paper]{ociamthesis}
% This one will format for PDF output (ie equal margins, no extra blank pages):
%\documentclass[a4paper,nobind]{templates/ociamthesis}
%UL 2 Dec 2018: pass this in from YAML
\documentclass[a4paper, nobind]{templates/ociamthesis}


% UL 30 Nov 2018 pandoc puts lists in 'tightlist' command when no space between bullet points in Rmd file
\providecommand{\tightlist}{%
  \setlength{\itemsep}{0pt}\setlength{\parskip}{0pt}}
 
% UL 1 Dec 2018, fix to include code in shaded environments

%UL 2 Dec 2018 reduce whitespace around verbatim environments
\usepackage{etoolbox}
\makeatletter
\preto{\@verbatim}{\topsep=0pt \partopsep=0pt }
\makeatother

%UL 26 Mar 2019, enable strikethrough
\usepackage[normalem]{ulem}

%%%%% SELECT YOUR DRAFT OPTIONS
% Three options going on here; use in any combination.  But remember to turn the first two off before
% generating a PDF to send to the printer!

% This adds a "DRAFT" footer to every normal page.  (The first page of each chapter is not a "normal" page.)
\fancyfoot[C]{\emph{DRAFT Printed on \today}}

% This highlights (in blue) corrections marked with (for words) \mccorrect{blah} or (for whole
% paragraphs) \begin{mccorrection} . . . \end{mccorrection}.  This can be useful for sending a PDF of
% your corrected thesis to your examiners for review.  Turn it off, and the blue disappears.

%%%%% BIBLIOGRAPHY SETUP
% Note that your bibliography will require some tweaking depending on your department, preferred format, etc.
% The options included below are just very basic "sciencey" and "humanitiesey" options to get started.
% If you've not used LaTeX before, I recommend reading a little about biblatex/biber and getting started with it.
% If you're already a LaTeX pro and are used to natbib or something, modify as necessary.
% Either way, you'll have to choose and configure an appropriate bibliography format...

% The science-type option: numerical in-text citation with references in order of appearance.
% \usepackage[style=numeric-comp, sorting=none, backend=biber, doi=false, isbn=false]{biblatex}
% \newcommand*{\bibtitle}{References}

% The humanities-type option: author-year in-text citation with an alphabetical works cited.
% \usepackage[style=authoryear, sorting=nyt, backend=biber, maxcitenames=2, useprefix, doi=false, isbn=false]{biblatex}
% \newcommand*{\bibtitle}{Works Cited}

%UL 3 Dec 2018: set this from YAML in index.Rmd
\usepackage[style=authoryear, sorting=nyt, backend=biber, maxcitenames=2, useprefix, doi=false, isbn=false]{biblatex}
\newcommand*{\bibtitle}{Works Cited}

% This makes the bibliography left-aligned (not 'justified') and slightly smaller font.
\renewcommand*{\bibfont}{\raggedright\small}

% Change this to the name of your .bib file (usually exported from a citation manager like Zotero or EndNote).
\addbibresource{bibliography/references.bib}


% Uncomment this if you want equation numbers per section (2.3.12), instead of per chapter (2.18):
%\numberwithin{equation}{subsection}


%%%%% THESIS / TITLE PAGE INFORMATION
% Everybody needs to complete the following:
\title{Exploring Social Metacognition:\\
the role of confidence in updating estimates of advisor reliability with and without feedback.}
\author{Matt Jaquiery}
\college{Wolfson College}

% Master's candidates who require the alternate title page (with candidate number and word count)
% must also un-comment and complete the following three lines:
%\masterssubmissiontrue
%\candidateno{933516}
%\wordcount{28,815}

% Uncomment the following line if your degree also includes exams (eg most masters):
%\renewcommand{\submittedtext}{Submitted in partial completion of the}
% Your full degree name.  (But remember that DPhils aren't "in" anything.  They're just DPhils.)
\degree{Doctor of Philosophy}
% Term and year of submission, or date if your board requires (eg most masters)
\degreedate{Michaelmas 2021}


%%%%% YOUR OWN PERSONAL MACROS
% This is a good place to dump your own LaTeX macros as they come up.

% To make text superscripts shortcuts
	\renewcommand{\th}{\textsuperscript{th}} % ex: I won 4\th place
	\newcommand{\nd}{\textsuperscript{nd}}
	\renewcommand{\st}{\textsuperscript{st}}
	\newcommand{\rd}{\textsuperscript{rd}}

%%%%% THE ACTUAL DOCUMENT STARTS HERE
\begin{document}

%%%%% CHOOSE YOUR LINE SPACING HERE
% This is the official option.  Use it for your submission copy and library copy:
\setlength{\textbaselineskip}{22pt plus2pt}
% This is closer spacing (about 1.5-spaced) that you might prefer for your personal copies:
%\setlength{\textbaselineskip}{18pt plus2pt minus1pt}

% You can set the spacing here for the roman-numbered pages (acknowledgements, table of contents, etc.)
\setlength{\frontmatterbaselineskip}{17pt plus1pt minus1pt}

% UL: You can set the line and paragraph spacing here for the separate abstract page to be handed in to Examination schools
\setlength{\abstractseparatelineskip}{13pt plus1pt minus1pt}
\setlength{\abstractseparateparskip}{0pt plus 1pt}

% UL: You can set the general paragraph spacing here - I've set it to 2pt (was 0) so
% it's less claustrophobic
\setlength{\parskip}{2pt plus 1pt}


% Leave this line alone; it gets things started for the real document.
\setlength{\baselineskip}{\textbaselineskip}


%%%%% CHOOSE YOUR SECTION NUMBERING DEPTH HERE
% You have two choices.  First, how far down are sections numbered?  (Below that, they're named but
% don't get numbers.)  Second, what level of section appears in the table of contents?  These don't have
% to match: you can have numbered sections that don't show up in the ToC, or unnumbered sections that
% do.  Throughout, 0 = chapter; 1 = section; 2 = subsection; 3 = subsubsection, 4 = paragraph...

% The level that gets a number:
\setcounter{secnumdepth}{2}
% The level that shows up in the ToC:
\setcounter{tocdepth}{2}


%%%%% ABSTRACT SEPARATE
% This is used to create the separate, one-page abstract that you are required to hand into the Exam
% Schools.  You can comment it out to generate a PDF for printing or whatnot.
\begin{abstractseparate}
  This \emph{R Markdown} template is for writing an Oxford University thesis. The template is built using Yihui Xie's \texttt{bookdown} package, with heavy inspiration from Chester Ismay's \texttt{thesisdown}, and the \texttt{OxThesis} \LaTeX~template (most recently adapted by John McManigle).
  
  This template's sample content include illustrations of how to do the various things you need to write a thesis in R Markdown, and largely follow the structure from \href{https://ulyngs.github.io/rmarkdown-workshop-2019/}{this R Markdown workshop}.
  
  Congratulations for taking a step further into the lands of open, reproducible science, by writing your thesis using a tool that allows you to transparently include tables and dynamically generated plots directly from the underlying data. Hip hooray!
\end{abstractseparate}

% JEM: Pages are roman numbered from here, though page numbers are invisible until ToC.  This is in
% keeping with most typesetting conventions.
\begin{romanpages}

% Title page is created here
\maketitle

%%%%% DEDICATION -- If you'd like one, un-comment the following.
\begin{dedication}
  For TBC
\end{dedication}

%%%%% ACKNOWLEDGEMENTS -- Nothing to do here except comment out if you don't want it.
\begin{acknowledgements}
 	This is where you will normally thank your advisor, colleagues, family and friends, as well as funding and institutional support. In our case, we will give our praises to the people who developed the ideas and tools that allow us to push open science a little step forward by writing plain-text, transparent, and reproducible theses in R Markdown.

We must be grateful to John Gruber for inventing the original version of Markdown, to John MacFarlane for creating Pandoc (\url{http://pandoc.org}) which converts Markdown to a large number of output formats, and to Yihui Xie for creating \texttt{knitr} which introduced R Markdown as a way of embedding code in Markdown documents, and \texttt{bookdown} which added tools for technical and longer-form writing.

Special thanks to \href{http://chester.rbind.io}{Chester Ismay}, who created the \texttt{thesisdown} package that helped many a PhD student write their theses in R Markdown. And a very special tahnks to John McManigle, whose adaption of Sam Evans' adaptation of Keith Gillow's original maths template for writing an Oxford University DPhil thesis in \LaTeX~provided the template that I adapted for R Markdown.

Finally, profuse thanks to JJ Allaire, the founder and CEO of \href{http://rstudio.com}{RStudio}, and Hadley Wickham, the mastermind of the tidyverse without whom we'd all just given up and done data science in Python instead. Thanks for making data science easier, more accessible, and more fun for us all.

\begin{flushright}
Ulrik Lyngs \\
Linacre College, Oxford \\
2 December 2018
\end{flushright}
\end{acknowledgements}


%%%%% ABSTRACT -- Nothing to do here except comment out if you don't want it.
\begin{abstract}
	This \emph{R Markdown} template is for writing an Oxford University thesis. The template is built using Yihui Xie's \texttt{bookdown} package, with heavy inspiration from Chester Ismay's \texttt{thesisdown}, and the \texttt{OxThesis} \LaTeX~template (most recently adapted by John McManigle).

This template's sample content include illustrations of how to do the various things you need to write a thesis in R Markdown, and largely follow the structure from \href{https://ulyngs.github.io/rmarkdown-workshop-2019/}{this R Markdown workshop}.

Congratulations for taking a step further into the lands of open, reproducible science, by writing your thesis using a tool that allows you to transparently include tables and dynamically generated plots directly from the underlying data. Hip hooray!
\end{abstract}

%%%%% MINI TABLES
% This lays the groundwork for per-chapter, mini tables of contents.  Comment the following line
% (and remove \minitoc from the chapter files) if you don't want this.  Un-comment either of the
% next two lines if you want a per-chapter list of figures or tables.
  \dominitoc % include a mini table of contents

% This aligns the bottom of the text of each page.  It generally makes things look better.
\flushbottom

% This is where the whole-document ToC appears:
\tableofcontents

\listoffigures
	\mtcaddchapter
  	% \mtcaddchapter is needed when adding a non-chapter (but chapter-like) entity to avoid confusing minitoc

% Uncomment to generate a list of tables:
\listoftables
  \mtcaddchapter
%%%%% LIST OF ABBREVIATIONS
% This example includes a list of abbreviations.  Look at text/abbreviations.tex to see how that file is
% formatted.  The template can handle any kind of list though, so this might be a good place for a
% glossary, etc.
\include{front-and-back-matter/abbreviations}

% The Roman pages, like the Roman Empire, must come to its inevitable close.
\end{romanpages}

%%%%% CHAPTERS
% Add or remove any chapters you'd like here, by file name (excluding '.tex'):
\flushbottom

% all your chapters and appendices will appear here
\hypertarget{part-introduction}{%
\part{Introduction}\label{part-introduction}}

\hypertarget{chapter-introduction}{%
\chapter{Introduction}\label{chapter-introduction}}

\adjustmtc

This thesis attempts to determine whether individual psychological processes in the seeking and taking of advice are sufficient to produce entrenched biases at the network level. The work includes empirical investigation into the individual psychological processes underlying advice seeking and advice taking using behavioural experiments; computational modelling and behavioural experimentation exploring the effects of contextual factors on advice-taking; agent-based computational modelling of the effects interactions between agents with psychological advisor evaluation processes that produce biases in assimilation of information and source selection; and a comparison between the structures of networks produced in these models and those naturally occurring social networks. The organisation is as follows: this introduction establishes the core concepts invoked in this thesis, and describes their treatment in the literature; the following sections each include a short chapter on the specific question addressed and the techniques used, a detailed description of the work conducted, and a short discussion of the conclusions drawn from the work; and a final section offers broader conclusions arising as a consequence of the presented work, alongside some suggestions for related research.

\hypertarget{overview}{%
\section*{Overview}\label{overview}}
\addcontentsline{toc}{section}{Overview}

The first line of behavioural experiments report investigates whether people decide which advisors to consult using internal signals where external feedback is unavailable, extending a previous finding concerning the influence of advice \autocite{pescetelliRoleDecisionConfidence2018} into the domain of advisor choice. While little support is found for the metacognitive mechanisms underpinning the influence of advice, the frequency of agreement is shown to be a good indicator of advisor choice. Next, the effects of advisor agreement are explored in terms of influence using a different decision-making domain; a date estimation rather than perceptual decision task. Again, people denied objective feedback on the quality of advice are shown to attend to agreement.

The work above demonstrates how the influence of advice depends upon the reputation of an advisor built up over time. In a second line of experiments, reputations of advisors are established at the outset and the responses to different pieces of advice are explored. People consider both the properties of advice and the properties of advisors, placing less trust in advice which appears suspicious or which comes from a suspicious source. These behavioural results support implications from evolutionary models which demonstrate that the presence of even a few bad actors in a population can mean that distrusting all advice from all sources a little is adaptive. The evolutionary models are extended to show that this kind of general distrust, known as egocentric discounting, emerges as adaptive even where none of the agents deliberately mislead others, and even where all agents are equally skilled and fully cooperative.

Finally, network effects arising from interconnected networks of agents are explored. Agents are based on the empirical results heretofore presented, and the consequences are explored for networks with various starting structures characterised by sparsity and homogeneity, under varying rates and reliabilities of external feedback. !TODO{[}results of this{]}. The resulting network characteristics are compared with network structures from real online networks from social media websites. !TODO{[}results of this{]}.

\hypertarget{advice}{%
\section*{Advice}\label{advice}}
\addcontentsline{toc}{section}{Advice}

Advice is broadly defined as information which comes from a social source. Advice is therefore different from other sources of information in that it is the result of (at least) mental processing of other information. In some cases, it may additionally include discussions among different group members (e.g.~advice from the International Advisory Panel on Climate Change). Throughout this thesis, the focus is primarily on advice which comes from a single, stable source, as when we see a post by an acquaintance on social media, or when a stranger provides us with advice.

\hypertarget{advice-taking}{%
\section*{Advice-taking}\label{advice-taking}}
\addcontentsline{toc}{section}{Advice-taking}

Advice occurs in the context of a decision, and forms a part of the information which is integrated during the decision-making process to produce a decision. To the extent that the decision reached differs from the decision that would have occurred had the advice not been presented, the advice has had an effect on the decision; to the extent that this difference changes the decision in a way consistent with the advice, the advice has been `taken' (as opposed to `rejected').

It is tacitly implied by many operationalizations of advice-taking that the informational content of the advice determines the extent to which it is taken or rejected \texttt{citation\ needed}. Insofar as the identity of the advisor matters, it matters because it functions as a cue to the informational content of the advice. This is likely a major oversimplification, however, because in many real-world contexts advice-giving and advice-taking form part of a developing social relationship: being consulted for advice and having one's advice followed are inherently rewarding \autocite{hertzIntrinsicValueSocial2018,hertzNeuralComputationsUnderpinning2017}; and taking advice can serve as a (sometimes costly) social signal of valuing a relationship with a person or group \autocite{byrneOstracismReducesReliance2016}. Furthermore, some authors have argued that people may perceive taking advice as sacrificing their independence or autonomy \autocite{CITATIONNEEDED}!TODO. While this thesis follows previous literature in omitting to consider the wider social concerns influencing the taking of advice, it is nevertheless important to remember that the processes investigated herein take place in a variety of social contexts where complex social agents attempt to optimise over numerous goals over numerous timescales.

\hypertarget{three-factor-model-of-trust}{%
\section*{Three-factor model of trust}\label{three-factor-model-of-trust}}
\addcontentsline{toc}{section}{Three-factor model of trust}

The degree to which advice is taken is proportional to the trust placed in the advisor by the decision-maker. Interpersonal trust, or the degree to which one is prepared to place one's fortune in the hands of another (e.g.~by relying on their advice), is apportioned by Mayer et al.~\autocite*{mayerIntegrativeModelOrganizational1995} onto three properties of the advisor (as judged by the decision-maker): ability, benevolence, and integrity. To these three properties of the advisor we may add the decision-maker's general propensity to trust, as well as situational cues and task cues (e.g.~the phenomenon that advice is more readily taken for hard tasks than easy ones, \autocite{ginoEffectsTaskDifficulty2007}).

\hypertarget{ability}{%
\subsection*{Ability}\label{ability}}
\addcontentsline{toc}{subsection}{Ability}

Ability captures the expertise of an advisor: their raw ability to perform the task for which they are giving advice. In some cases this is relatively straightforward, as in the expertise of a General Practitioner in matters of health and disease, and in others more complex, as in the expertise of a hairdresser when deciding on a haircut (when matters of personal taste comingle with aesthetic considerations of facial structure, practical considerations of hair constitution, and social considerations of fashion). The greater the ability of an advisor, the greater the influence of their advice, as demonstrated by experiments showing that participants' decisions are more affected by the advice of advisors who are labelled as more expert in a relevant domain \autocite{sahCheapTalkCredibility2013,schultzeInabilityIgnoreUseless2017,sniezekImprovingJudgementPrepaid2004,sniezekTrustConfidenceExpertise2001,sollJudgmentalAggregationStrategies2011}, or are shown to be more expert empirically \autocite{pescetelliUseMetacognitiveSignals2017,sahCheapTalkCredibility2013,yanivAdviceTakingDecision2000}.

\hypertarget{benevolence}{%
\subsection*{Benevolence}\label{benevolence}}
\addcontentsline{toc}{subsection}{Benevolence}

Benevolence refers to the extent to which the advisor seeks to further the interests of the decision-maker. Where ability represents the absolute limit on the quality of advice, benevolence represents the extent to which the advice approaches this limit. The advice of even a renouned expert may be doubted if there is reason to believe their goal is to mislead, a vital lesson for medieval monarchs with their councils of politicking advisors. Experimental work has shown that psychology students relied more on the advice of their friends than on the advice of labelled experts \autocite{CITATIONNEEDED}!TODO, and that participants are more inclined to reject advice when uncertainty is attributed to malice rather than ignorance \autocite{schulInfluencesDistrustTrust2015}.

\hypertarget{integrity}{%
\subsection*{Integrity}\label{integrity}}
\addcontentsline{toc}{subsection}{Integrity}

Advisors with integrity exhibit adherence to principles which the decision-maker endorses. As with benevolence, integrity acts to determine the extent to which advice approaches the limit imposed by ability. While not mutually exclusive, integrity is typically important where relationships are less personal (e.g.~we may place great trust in a General Practitioner because of their expertise in medical matters and their \emph{integrity} in adhering to a set of professional ethical and conduct requirements). \texttt{Some\ description\ of\ the\ research}\autocite{CITATIONNEEDED}!TODO

\hypertarget{normative-models-of-advice-taking}{%
\section*{Normative models of advice-taking}\label{normative-models-of-advice-taking}}
\addcontentsline{toc}{section}{Normative models of advice-taking}

Advice-taking can be evaluated formally with reference to a normative model. The simplest and most common of these views the decision-making task as an estimation problem (or combination of estimation problems), and provides an approximately Bayesian variance-weighted integration of independent estimates. To borrow from Galton \autocite*{galtonVoxPopuliWisdom1907}, consider the task of judging the weight of a bullock. We can model any single guess (\(e\)) as the true weight (\(v\)) plus some error (\(\epsilon\)):

\begin{equation}
e = v + \epsilon
\label{eq:estimate}
\end{equation}

The key insight is to observe that the error is drawn from a normal distribution (\(\mathcal{N}(\mu=0,\,\sigma^{2})\))\footnote{The normal distribution is well-supported by empirical evidence, but note that any symmetrical distribution around 0 will lead to the same conclusion.}. As the number of samples from this distribution increases, the mean of those samples tends towards the mean of the distribution. Thus, the more estimates are taken, the closer on average the sum of errors will be to 0.

\begin{align}
\frac{\sum_{i}^{N}(e_{i})}{N} &= \frac{\sum_{i}^{N}(v + \mathcal{N}(\mu=0,\,\sigma^{2}_{i}))}{N}\\
\frac{\sum_{i}^{N}(e_{i})}{N} &= \frac{\sum_{i}^{N}(v)}{N} + \frac{\sum_{i}^{N}(\mathcal{N}(\mu=0,\,\sigma^{2}_{i}))}{N}\\
\frac{\sum_{i}^{N}(e_{i})}{N} &= \frac{Nv}{N} + \hat{0}\\
\frac{\sum_{i}^{N}(e_{i})}{N} &\approx v
\label{eq:wisdom-of-crowds}
\end{align}

Observe that this formulation is true no matter the value of \(N\). On average, it is always better to have more estimates than fewer. This suggests that, even in the situation where there are only two estimates (the decision-maker's and the advisor's), the best policy will be to incorporate both estimates into the final decision.

The variance of the normal distribution from which errors are derived (\(\sigma^{2}_{i}\)) is, in the example above, drawn from a normal distribution itself (\(\sigma^{2}_{i} \sim \mathcal{N}(\mu=0,\,\text{sd}^{2})\)meaning that it is also cancelled out on average over repeated samples). Where few estimates are taken, weighting those estimates by the variance of the error distributions will increase the accuracy of the estimates in proportion to the difference between the variances. Many experimental implementations of this model avoid weighting issues by calibrating decision-makers and advisors to be equally accurate on average (\(\sigma^{2}_{\text{decision-maker}}=\sigma^{2}_{\text{advisor}}\)). The result of this constraint is that the optimal policy is simply to average all estimates together.

\hypertarget{egocentric-discounting}{%
\section*{Egocentric-discounting}\label{egocentric-discounting}}
\addcontentsline{toc}{section}{Egocentric-discounting}

From the perspective of the \protect\hyperlink{normative-models-of-advice-taking}{normative model above}, decision-makers should weigh their own estimate equally with each other estimate they receive in the process of coming to their decision. One of the most robust findings in the literature on advice-taking is that people routinely underweight advisory estimates relative to their own estimates, a phenomenon known as \emph{egocentric discounting}\autocite{danaAdviceChoice2015,ginoEffectsTaskDifficulty2007,hutterSeekingAdviceSampling2016,libermanNaiveRealismCapturing2012,minsonCostCollaborationWhy2012,raderAdviceFormSocial2017,ronayneIgnoringGoodAdvice2018,seeDetrimentalEffectsPower2011,sollJudgmentalAggregationStrategies2011,troucheVigilantConservatismEvaluating2018,yanivAdviceTakingDecision2000,yanivExploitingWisdomOthers2012,yanivUsingAdviceMultiple2007}. Egocentric discounting occurs in both feedback and no-feedback contexts \autocite{yanivAdviceTakingDecision2000}.

Explanations for egocentric discounting are usually framed in terms of personal-level psychology: decision-makers have better access to reasons for their decision \autocite{yanivAdviceTakingDecision2000}; overrate their own competence \autocite{sniezekImprovingJudgementPrepaid2004}; may have a desire to appear consistent \autocite{yanivUsingAdviceMultiple2007}; may see opinions as possessions \autocite{sollJudgmentalAggregationStrategies2011}; may be loss-averse to providing a worse final estimate due to advice-taking \autocite{sollJudgmentalAggregationStrategies2011}; or have difficulty avoiding anchoring \autocite{schultzeInabilityIgnoreUseless2017} or repetition bias effects \autocite{troucheVigilantConservatismEvaluating2018}. None of these explanations has survived rigorous empirical testing, however, and recently suggestions have widened to include consideration of aggregate-level rather than personal-level causes, with Trouche et al.~\autocite*{troucheVigilantConservatismEvaluating2018} arguing that the potential for misaligned incentives between decision-maker and advisor motivate discounting of advice.

In the course of this thesis \texttt{crossref\ needed}, I demonstrate that egocentric discounting may be a stable metastrategy which protects against exploitation, carelessness, incompetence, and miscommunication. From this perspective, the normative model pertains to a particular instantiation of a problem with questionable ecological validity given the typical ethology of advice-taking in humans. While such considerations affect the conclusions one draws from egocentric discounting relative to the normative model, they do not detract substantially from the practice of using the normative model as an optimum `set point' from which to evaluate advice-taking behaviour.

\hypertarget{evaluation-of-advice}{%
\section*{Evaluation of advice}\label{evaluation-of-advice}}
\addcontentsline{toc}{section}{Evaluation of advice}

The value of advice, in informational terms, is measured in relation to the optimal decision. It is necessary to distinguish two perspectives on the value of advice: the optimality of the decision recommended by the advice itself; and the optimality of the decision based on advice relative to the decision which would have been made had the advice not been received. This distinction is necessary because people alter their advice-giving behaviour in anticipation of discounting on the part of the decision-maker \autocites{CITATIONNEEDED}[for a case in human-machine teaming see][]{azariaStrategicAdviceProvision2016}!TODO, somewhat akin to starting negotiations with a higher demand than one is hoping to settle for.

A single piece of advice can be evaluated using its own properties and the properties of the advisor giving the advice. Furthermore, that evaluation can serve to update the properties of the advisor. A piece of advice's own properties will include its plausibility (e.g.~participants in estimation tasks discount advice which is distant from their own initial estimates more heavily \autocite{yanivReceivingOtherPeople2004}), while the properties of the advisor will include the advisor's trustworthiness (\protect\hyperlink{three-factor-model-of-trust}{see above}). The updating of trust following experience of advice is likely to be largely in the domain of \protect\hyperlink{ability}{ability}, although other domains may be affected where the advice is particularly egregious.

\hypertarget{updating-advisor-evaluations}{%
\section*{Updating advisor evaluations}\label{updating-advisor-evaluations}}
\addcontentsline{toc}{section}{Updating advisor evaluations}

While a single piece of advice must be taken on its own terms, people can construct relatively accurate estimates of advisors' advice when provided with feedback on the decisions they use the advice to make \autocite{pescetelliUseMetacognitiveSignals2017,sahCheapTalkCredibility2013,yanivAdviceTakingDecision2000}. This likely happens as an analogue of reinforcement learning, where feedback allows an error signal to be used to update the estimate of the advisor's ability (\(\hat{s}^{a}\)) rather than one's own beliefs about the world, according to some learning rate (\(\lambda\)).

\begin{align}
\text{this is wrong; need to check RL models for an analogue}\\
\hat{s}_{t+1} = \hat{s}_{t} + |e^{a}_{t} - v|\cdot\lambda
\label{eq:advisor-evaluation}
\end{align}

\hypertarget{advisor-evaluation-without-feedback}{%
\section*{Advisor evaluation without feedback}\label{advisor-evaluation-without-feedback}}
\addcontentsline{toc}{section}{Advisor evaluation without feedback}

Where feedback is not available, participants in experiments continue to demonstrate an ability to respond rationally to differences in advisor quality \autocite{pescetelliUseMetacognitiveSignals2017}. This is evidently not done through access to the correct real-world values, because feedback providing those values is unavailable, and, were participants aware of those values themselves, it stands to reason they would have provided those values (and thus not require advice!). Pescetelli and Yeung \autocite*{pescetelliUseMetacognitiveSignals2017} suggest the mechanism for this ability to discriminate between advisors in the absence of feedback is performing updates based on confidence-weighted agreement.

\hypertarget{agreement}{%
\subsection*{Agreement}\label{agreement}}
\addcontentsline{toc}{subsection}{Agreement}

Consider first the non-weighted agreement case, where the advisor's estimate at time \(t\) (\(e^{a}_{t}\)) and the decision-maker's estimates (\(e^d\)) are binary (\(\in0,1\)). The estimate of the advisor's ability (\(\hat{s}^{a}\)) is updated positively if the advisor and decision-maker agree, and negatively otherwise, according to the learning rate \(\lambda\).

\begin{align}
\hat{s}^a_{t+1} = \hat{s}^a_{t} + (-2 \text{ } |e^{d}_{t}-e^{a}_{t}|+1)\cdot\lambda
\label{eq:advisor-agreement-binary}
\end{align}

\hypertarget{confidence-weighted-agreement}{%
\subsubsection*{Confidence-weighted agreement}\label{confidence-weighted-agreement}}
\addcontentsline{toc}{subsubsection}{Confidence-weighted agreement}

The updating of advice contingent on agreement may be weighted by confidence in the initial decision (\(c^d\)), such that agreement and disagreement are considered more informative about the quality of the advice when the decision with which they agree or disagree is more certain.

\begin{align}
\hat{s}^a_{t+1} = \hat{s}^a_{t} + (-2 \text{ } |e^{d}_{t}-e^{a}_{t}|+1)\cdot c^d_t \cdot \lambda
\label{eq:advisor-agreement-weighted-binary}
\end{align}

\hypertarget{continuous-estimate-case}{%
\subsubsection*{Continuous estimate case}\label{continuous-estimate-case}}
\addcontentsline{toc}{subsubsection}{Continuous estimate case}

\hypertarget{homophily-and-echo-chambers}{%
\section*{Homophily and echo-chambers}\label{homophily-and-echo-chambers}}
\addcontentsline{toc}{section}{Homophily and echo-chambers}

Homophily is the ubiquitous phenomenon that individuals more closely connected to one another within a social network tend to be more similar to one another than would be expected by chance across numerous dimensions, from demographics to attitudes \autocite{mcphersonBirdsFeatherHomophily2001}. Whether homophily in virtual social networks is responsible for increases in polarisation is debated. Proponents point to increases in polarisation (e.g.~in politics: \texttt{Pew\ Research\ Center,\ 2014}!TODO), to empirical studies demonstrating homophily in virtual social networks \autocite{cardosoTopicalHomophilyOnline2017,colleoniEchoChamberPublic2014}, and to studies examining selective exposure online \autocite{kobayashiSELECTIVEEXPOSUREPOLITICAL2009}, and to echo chambers: egregious examples of highly homophilous networks with pathological polarisation \autocite{sunsteinRepublicCom2002,sunsteinRepublicDividedDemocracy2018}. The empirical components of the argument are contested, with evidence that virtual social networks are less homogenous than offline social networks \autocite[and hence depolarising,][]{barberaHowSocialMedia2015}, and that selective exposure is a somewhat dubious finding which does not show up clearly online \autocite{garrettEchoChambersOnline2009,garrettPoliticallyMotivatedReinforcement2009,nelsonMythPartisanSelective2017,searsSelectiveExposureInformation1967}. Modelling work demonstrates, however, that where there is a bias in assimilation of information, homophily exacerbates polarisation \autocite{dandekarBiasedAssimilationHomophily2013}. Where polarisation in turn increases homophily, for example through selective exposure or avoidance, a self-reinforcing spiral emerges wherein social connections become increasingly homogenous and attitudes increasingly extreme \autocite{songDynamicSpiralsPut2017}.

\hypertarget{source-selection-and-information-weighting}{%
\section*{Source selection and information weighting}\label{source-selection-and-information-weighting}}
\addcontentsline{toc}{section}{Source selection and information weighting}

\hypertarget{context-dependency-of-epistemic-processes}{%
\section*{Context-dependency of epistemic processes}\label{context-dependency-of-epistemic-processes}}
\addcontentsline{toc}{section}{Context-dependency of epistemic processes}


%%%%% REFERENCES

% JEM: Quote for the top of references (just like a chapter quote if you're using them).  Comment to skip.
% \begin{savequote}[8cm]
% The first kind of intellectual and artistic personality belongs to the hedgehogs, the second to the foxes \dots
%   \qauthor{--- Sir Isaiah Berlin \cite{berlin_hedgehog_2013}}
% \end{savequote}

\setlength{\baselineskip}{0pt} % JEM: Single-space References

{\renewcommand*\MakeUppercase[1]{#1}%
\printbibliography[heading=bibintoc,title={\bibtitle}]}

\end{document}
